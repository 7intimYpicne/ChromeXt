% ! TEX program = lualatex
\documentclass[aspectratio=169]{beamer}

\usepackage{emoji}

\usefonttheme[onlymath]{serif}
\beamertemplatenavigationsymbolsempty
\setbeamertemplate{footline}[frame number]{}
\setbeamertemplate{caption}{\raggedright\insertcaption\par}
\newcommand{\diff}{\operatorname{d}}
\newcommand{\info}[1]{\texttt{{\color{teal}(#1)}}}

\title{
Introduction to ChromeXt\\
{\color{teal}\normalsize UserScript and DevTools supports for mobile browsers }\\
}
\author{\texttt{JingMatrix}}
\date{\small \color{gray} \today}

\begin{document}
{
\setbeamertemplate{footline}{}
\begin{frame}[noframenumbering]
	\titlepage
\end{frame}
}

\section{Background}

\setbeamercovered{transparent}
\begin{frame}
	\frametitle{What is ChromeXt?}
	\begin{definition}
		ChromeXt is {\color<2-5>{brown} a tiny Xposed module}
		\info{Android} that adds {\color<6-9>{brown} UserScript}, {\color<10-12>{brown} DevTools} and
		some other functions to {\color<13-15>{brown} Chromium based and WebView based mobile browsers}.
	\end{definition}

	\vspace{0.5cm}
	\uncover<2->{
		\begin{block}{{\only<2-5>{What is Xposed}\only<6-9>{What are UserScripts}\only<10-12>{What are DevTools}\only<13-15>{Which browsers are supported}}? }
			\only<2-15>{
				\begin{itemize}
					\only<2-5>{
					\item<2-> Xposed is a framework helping developers to modify Android applications.
					\item<3-> The most popular Xposed framework for Android $8.1 \sim 14$ is LSPosed.
					\item<4-> ChromeXt is an Xposed module with size less than $0.1$\,MB.
					\item<5-> Both ChromeXt and LSPosed are open source projects powered
						by \only<5>{\emoji{heart}}.
						}

						\only<6-9>{
					\item<6-> UserScripts enable users to modify websites in their browsers.
					\item<7-> UserScripts are supported by most desktop browsers but only few mobile browsers.
					\item<8-> Popular UserScript managers include Tampermonkey and Violentmonkey;
						and people usually share their UserScripts on GreasyFork or GitHub.
					\item<9-> ChromeXt works as a UserScript manager for mobile browsers.
						}
						\only<10-12>{
					\item<10-> DevTools help developers to inspect, modify and debug websites in browsers.
					\item<11-> They are the primary tools for hacking or finding vulnerabilities of websites.
					\item<12-> Usually one needs desktop browsers to inspect websites in mobile phones.
						}
						\only<13-15>{
					\item<13-> Chromium based browsers include Chrome, Edge, Bromite, Brave,
						Vivalid $\ldots$
					\item<14-> WebView based browsers include Via, Soul, FOSS Browser $\ldots$
					\item<15-> Not supported: Firefox, Samsung Internet Browser, Opera.
						}
				\end{itemize}
			}
		\end{block}
	}
\end{frame}

\section{Installation and Usage}
\begin{frame}
	\frametitle{How to install ChromeXt}
	\begin{block}{Root users}
		Install LSPosed \info{Magisk module}
		and then install ChromeXt.
	\end{block}
	\vspace{1cm}
	\uncover<2->{
		\begin{block}{Non-root users}
			\begin{enumerate}
				\item<2-> Download LSPatch \info{modified by JingMatrix} and ChromeXt.
				\item<3-> If Java is available, then use lspatch.jar, otherwise install manager.apk.
				\item<4-> Patch the target browser to embed ChromeXt.apk.
			\end{enumerate}
		\end{block}
	}
\end{frame}

\begin{frame}
	\frametitle{How to use ChromeXt?}
	ChromeXt is fully integrated into the target browser, almost all interactions are
	done within the browser.

	\uncover<2->{
		\begin{block}{Different ways to install UserScripts}
			Open .user.js URLs, open local UserScripts with ChromeXt, import via
			Eruda console.
		\end{block}
	}

	\uncover<3->{
		\begin{block}{Functions offered by ChromeXt}
			\begin{itemize}
				\item<3-> Via front end: manage and modify installed UserScripts
				\item<4-> Via page menu: reader mode, Eruda console and Developer Tools
				\item<5-> Via Developer options setting: set gesture navigation, export bookmarks
				\item<6-> Via Eruda console: cosmetic filters, user-agent spoofing, UserScript commands
			\end{itemize}
		\end{block}
	}
\end{frame}

\section{Tutorial to hack YouTube services}

\begin{frame}
	\frametitle{Tutorial: write a UserScript to remove YouTube advertisements}
	\begin{block}{Analysis}
		\begin{enumerate}
			\item<1-> Different videos contains different advertisements to be played at
				different time, but the YouTube page does not reload when we switch videos.
			\item<2-> Therefore, advertisement data are fetched from remote for each new video.
			\item<3-> Find the code that fetching remote advertisement data.
			\item<4-> Change the fetched data to clear all advertisement data.
		\end{enumerate}
	\end{block}

	\uncover<5->{
		\begin{block}{Start writing a UserScript}
			\begin{enumerate}
				\item Only partial code will be shown for instructive purpose.
				\item The previously described method is novel, no source code available online.
				\item YouTube Music has the same vulnerability.
			\end{enumerate}
		\end{block}
	}
\end{frame}

\end{document}
